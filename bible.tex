\documentclass[
  twoside,
%  draft,
  ]{bible}

\usepackage{metalogo} % For Xe(La)TeX Logos

\usepackage{microtype}

\def\today{March 2, 2011}
\title{The King James Bible}

\renewcommand{\thefootnote}{\alph{footnote}}

\def\usecs#1{\csname#1\endcsname}

\setlength{\parindent}{1em}

\BibleVersion{kjv}
\begin{document}
\NewSpeaker{\Father}{blue}{blue!50!black}
\NewSpeaker{\Son}{red}{red!50!black}

\input{hyphenations}

%\input{notes}

\null\vfil
\centerline{The Bible\TeX\ Project}
\vfil
%\testbook{old/genesis}

\clearpage

\onecolumn
\thispagestyle{empty}
\null\vfil
\centerline{\Large The Old Testament of the King~James~Version of the Bible}
\vfil
\twocolumn


%\usecs{note genesis 1:1}
%\usecs{note genesis 1:2}
%\usecs{note genesis 3:16}
%\usecs{note exodus 1:1}

%\Huge


\InputBook{name = The First Book of Moses, toc-name = Genesis}{old/genesis}
\InputBook{name = The Second Book of Moses, toc-name = Exodus}{old/exodus}
\InputBook{name = The Third Book of Moses, toc-name = Leviticus}{old/leviticus}
\InputBook{name = The Fourth Book of Moses, toc-name = Numbers}{old/numbers}
\InputBook{name = The Fifth Book of Moses, toc-name = Deuteronomy}{old/dueteronomy}
\InputBook{name = Joshua}{old/joshua}
\InputBook{name = Judges}{old/judges}
\InputBook{name = Ruth}{old/ruth}
\InputBook{name = The First Book of Samuel, toc-name = Samuel I}{old/samuel1} % KINGS I
\InputBook{name = The Second Book of Samuel, toc-name = Samuel II}{old/samuel2} % KINGS II
\InputBook{name = The First Book of the Kings, toc-name = Kings I}{old/kings1} % KINGS III
\InputBook{name = The Second Book of the Kings, toc-name = Kings II}{old/kings2} % KINGS IV
\InputBook{name = The First Book of the Chronicles, toc-name = Chronicles I}{old/chronicles1}
\InputBook{name = The Second Book of the Chronicles, toc-name = Chronicles II}{old/chronicles2}
\InputBook{name = Ezra}{old/ezra}
\InputBook{name = Nehemiah}{old/nehemiah}
\InputBook{name = Esther}{old/esther}
\InputBook{name = Job}{old/job}
\InputBook{name = Psalms}{old/psalms}
\InputBook{name = Proverbs}{old/proverbs}
\InputBook{name = The Preacher, toc-name = Ecclesiastes}{old/ecclesiastes}
\InputBook{name = The Song of Solomon}{old/songofsolomon}
\InputBook{name = The Book of the Prophet Isaiah, toc-name = Isaiah}{old/isaiah}
\InputBook{name = The Book of the Prophet Jeremiah, toc-name = Jeremiah}{old/jeremiah}
\InputBook{name = The Lamentations of Jeremiah, toc-name = Jeremiah}{old/lamentations}
\InputBook{name = The Book of the Prophet Ezekiel, toc-name = Exekiel}{old/ezekiel}
\InputBook{name = Daniel}{old/daniel}
\InputBook{name = Hosea}{old/hosea}
\InputBook{name = Joel}{old/joel}
\InputBook{name = Amos}{old/amos}
\InputBook{name = Obadiah}{old/obadiah}
\InputBook{name = Jonah}{old/jonah}
\InputBook{name = Micah}{old/micah}
\InputBook{name = Nahum}{old/nahum}
\InputBook{name = Habakkuk}{old/habakkuk}
\InputBook{name = Zechariah}{old/zechariah}
\InputBook{name = Malachi}{old/malachi}


\onecolumn
\thispagestyle{empty}
\null\vfil
\centerline{\Large The New Testament of the King~James~Version of the Bible}
\vfil
\twocolumn

\InputBook{toc-name = Matthew,           name = The Gospel According to Saint Matthew}{new/matthew}
\InputBook{toc-name = Mark,              name = The Gospel According to Saint Mark}{new/mark}
\InputBook{toc-name = Luke,              name = The Gospel According to Saint Luke}{new/luke}
\InputBook{toc-name = John,              name = The Gospel According to Saint John}{new/john}
\InputBook{toc-name = Acts,              name = The Acts of the Apostles}{new/acts}
\InputBook{toc-name = Romans,            name = The Epistle of Paul the Apostle to the Romans}{new/romans}
\InputBook{toc-name = I Corinthians,     name = The First Epistle of Paul the Apostle to the Corinthians}{new/corinthians1}
\InputBook{toc-name = II Corinthians,    name = The Second Epistle of Paul the Apostle to the Corinthians}{new/corinthians2}
\InputBook{toc-name = Galatians,         name = The Epistle of Paul the Apostle to the Galatians}{new/galatians}
\InputBook{toc-name = Ephesians,         name = The Epistle of Paul the Apostle to the Ephesians}{new/ephesians}
\InputBook{toc-name = Philippians,       name = The Epistle of Paul the Apostle to the Philippians}{new/philippians}
\InputBook{toc-name = Colossians,        name = The Epistle of Paul the Apostle to the Colossians}{new/colossians}
\InputBook{toc-name = Thessalonians I,   name = The First Epistle of Paul the Apostle to the Thessalonians}{new/thessalonians1}
\InputBook{toc-name = Thessalonians II,  name = The Second Epistle of Paul the Apostle to the Thessalonians}{new/thessalonians2}
\InputBook{toc-name = Timothy I,         name = The First Epistle of Paul the Apostle to Timothy}{new/timothy1}\InputBook{toc-name = Timothy II} name = The Second Epistle of Paul the Apostle to Timothy,{new/timothy2}\InputBook{toc-name = Titus} name = The Epistle of Paul the Apostle to Titus,{new/titus}\InputBook{toc-name = Philemon, name = The Epistle of Paul the Apostle to Philemon}{new/philemon}
\InputBook{toc-name = Hebrews,           name = The Epistle of Paul the Apostle to the Hebrews}{new/hebrews}
\InputBook{toc-name = James,             name = The General Epistle of James}{new/james}
\InputBook{toc-name = Peter,             name = The First Epistle General of Peter I}{new/peter1}
\InputBook{toc-name = Peter,             name = The Second General Epistle of Peter II}{new/peter2}
\InputBook{toc-name = I John,            name = The First Epistle General of John I}{new/john1}
\InputBook{toc-name = John,              name = The Second Epistle General of John II}{new/john2}
\InputBook{toc-name = John,              name = The Third Epistle General of John III}{new/john3}
\InputBook{toc-name = Jude,              name = The General Epistle of Jude}{new/jude}
\InputBook{toc-name = Revelations,       name = The Revelation of Saint~John~the~Devine}{new/revelations}

\backmatter\onecolumn
% Need to make single column
\include{\bibleversion/license-page}

% colophon

\newgeometry{centering,width=4in}

\thispagestyle{empty}
\null\vfil\noindent
This King~James~Bible was typeset by \XeTeX\ (using the \XeLaTeX\ format) in
6.64~seconds on a single thread of an Intel~\mbox{i7-3770K} running at
$3.50\,\mathrm{GHz}$.  \XeLaTeX\ is based upon the original \TeX\
typesetting system by Dr.~Donald~E.~Knuth and its extensions (\LaTeX)
written by Leslie~Lamport.
\vfil\vfil
\end{document}

%%% Local Variables:
%%% mode: latex
%%% TeX-master: t
%%% TeX-engine: default
%%% End:
